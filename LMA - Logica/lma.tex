\documentclass{article}

\usepackage[margin=1in]{geometry}

\author{Vinícius Hansen}
\title{Trabalho Escrito de Lógica Matemática \\ Tema: Criptomoedas}
\date{}

\begin{document}

\maketitle

\section{Tema}
Nesse trabalho serão abordados assuntos relacionados ao mercado de criptomoedas.


\section{Frases}
\begin{enumerate}
    \item Existem criptomoedas que são golpe
    \item Bitcoin é uma criptomoeda
    \item Toda criptomoeda é descentralizada
    \item Existem criptomoedas que não são sigilosas
    \item O valor do Bitcoin é de 40.000 dólares por unidade
    \item Criptomoedas são armazenadas em carteiras digitais ou físicas
    \item Nenhuma criptomoeda é material
    \item Algumas criptomoedas são rápidas e seguras
    \item Bitcoin é descentralizado
    \item Qualquer pessoa que possua a chave privada da carteira tem controle total da carteira
\end{enumerate}


\section{Fórmulas, predicados e funções}

\subsection{Definição de predicados}
\begin{itemize}
    \item \emph{Golpe(x): x} é golpe
    \item \emph{Desc(x): x} é descentralizado(a)
    \item \emph{Cripto(x): x} é uma criptomoeda
    \item \emph{Sigiloso(x): x} é sigiloso(a)
    \item \emph{Armazenado(x,y): x} está armazenado em $y$
    \item \emph{Rápido(x): x} é rápido
    \item \emph{Seguro(x): x} é seguro
    \item \emph{Material (x): x} é material
    \item \emph{Possui (x,y):} a pessoa $x$ possuí $y$
    \item \emph{Total(x, y)} a pessoa $x$ tem controle total da carteira $y$
\end{itemize}


\subsection{Definição de Funções}
\begin{itemize}
    \item \emph{valor(x):} retorna o valor de uma unidade da criptomoeda \emph{x}
\end{itemize}

\subsection{Fórmulas}

\begin{enumerate}
    \item $\exists$ \emph{c} G(\emph{c})
    \item $\forall$ \emph{b} C(\emph{b})
    \item $\forall$ \emph{c} D(\emph{c})
    \item $\exists$ \emph{c} $\neg$ S(\emph{c})
    \item $\forall$ \emph{b} valor(\emph{b}) = 40.000 US\$
    \item $\forall$ \emph{c} $\exists$ \emph{d} $\exists$ \emph{f} (A(\emph{c,d}) $\lor$ A(\emph{c,f}))
    \item $\neg$ $\exists$ \emph{c} M(\emph{c})
    \item $\exists$ \emph{c} (R(\emph{c}) $\land$ S(\emph{c}))
    \item $\forall$ \emph{b} (C(\emph{b}) $\Rightarrow$ D(\emph{b}))
    \item $\forall$ \emph{p} $\exists$ \emph{w} $\exists$ \emph{k} (P(\emph{$p,k$}) $\Rightarrow$ T($p,w$))
\end{enumerate}
\section{Assinatura}
$\sum$ ({\bf Con}, {\bf Fun}, {\bf Pred})

{\bf Con} = \{$c,b,d,f,p,k,w$\}

{\bf Fun} = \{$valor(x)$\}

{\bf Pred = }\{Golpe^{1}, Desc$¹$, Cripto$¹$, Sigiloso$¹$, Armazenado$²$, Rápido$¹$, Seguro$¹$, Material$¹$, Possui$²$, Total$²$\}

\subsection{Termos}
\begin{itemize}
    \item $c$
    \item valor($x$)
    \item $f$
    \item $w$
\end{itemize}









\end{document}
